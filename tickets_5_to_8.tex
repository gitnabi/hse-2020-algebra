%%%%%%%%%%%%%%%%%%%%%%%%%%%%%%%%%%%%%%%%%%%%
%%%%%%%%%%%%%%%%%%%%%%%%%%%%%%%%%%%%%%%%%%%%
%%%%%%%%%%%%%%%%%%%%%%%%%%%%%%%%%%%%%%%%%%%% 5
\mysection
По аналогии с отношением $L_H$ на группе $G$ можно определить другое отношение
$R_H$ следующим образом: $(a, b) \in R_H \eq ba^{-1} \in H$.
Совершенно аналогично показывается, что $R_H$ --- тоже отношение
эквивалентности на $G$ и что классом элемента $a$ будет множество
$Ha := \left\{ ha \mid h \in H \right\}$, называется 
\textit{правым смежным классом} элемента $a$.

Для нейтрального элемента $e \in G$ имеем $eH = H = He$, так что левый и
правые смежные классы для $e$ равны между собой и совпадают с $H$. 
Подчеркнем однако, что в общем случае для одного и того же элемента 
$a \in G$ смежные классы $aH$ и $Ha$ 
вполне могут оказаться разными множествами даже несмотря на то, 
что сам элемент $a$ принадлежит каждому из них.

Таким образом, мы выяснили, что, с одной стороны, группа $G$ 
разбивается в объединение попарно непересекающихся левых смежных классов,
а с другой стороны, $G$ разбивается в объединение попарно 
непересекающихся левых смежных классов.
Вообще говоря, это \textbf{два разных разбиения}.
A вот ситуация, когда эти два разбиения совпадают, заметно выделяется среди
остальных наличием замечательных свойств, которые мы сейчас и обсудим.

\subsection{Нормальные подгруппы.}


\begin{definition}
    Подгруппа $H \subseteq G$ называется \textit{нормальной},
    если $gH = Hg$ для всех $g \in G$.
\end{definition}

Обозначение: $H \triangleleft G$.

\begin{example}[ы] \vspace{-0.3cm}
    \begin{enumerate}[label = \arabic*)]
        \item $G$ абелева $\To$ всякая подгруппа в $G$ автоматически нормальна.
        \item $G = S_3, H = \left\{ id, (12) \right\} \To H$
        не является нормальной подгруппой в $G$.
        \item $H = \left\{ e \right\}$ или $H = G$ 
        (то есть $H$ --- несобственная подгруппа) $\To H \triangleleft G$.
    \end{enumerate}
\end{example}

Следующее предложение дает несколько эквивалентных условий,
определяющих нормальную подгруппу.

\begin{proposal}
    Пусть $H$ --- подгруппа группы $G$. Тогда следующие условия эквивалентны:
    \vspace{-0.2cm}
    \begin{enumerate}[label=(\arabic*)]
        \item $H \triangleleft G$;
        \item $gHg^{-1} = H \ \forall g \in G$;
        \item $gHg^{-1} \subseteq H \ \forall g \in G$.
    \end{enumerate}
\end{proposal}

\begin{proof}
    (1)$\Rightarrow$(2) Если $gH = Hg$, то, умножая обе части на $g^{-1}$ справа,
    получаем $gHg^{-1} = H$.

    (2)$\Rightarrow$(3) Тривиально.

    (3)$\Rightarrow$(1) Пусть $gHg^{-1} \subseteq H$.
    Умножая обе части на $g^{-1}$ справа, получаем $gH \subseteq Hg$.
    Поскольку условие (3) верно для любого $g \in G$, то оно останется
    верным после замены в нем $g$ на $g^{-1}$, так что $g^{-1}Hg \subseteq H$.
    Умножая обе части последнего включения на $g$ слева, получаем 
    $Hg \subseteq gH$. Значит, $gH = Hg$.
\end{proof}

\subsection{Факторгруппы.}

Пусть $H$ --- нормальная подгруппа группы $G$. Согласно определению,
в этой ситуации левые и правые смежные классы $G$ по $H$ --- это одно и то же,
и тогда мы будем называть их просто смежными классами.

Обозначим через $G / H$ множество всех смежных классов $G$ по $H$.
Оказывается, что на $G / H$ можно ввести структуру группы.

Сначала введем на $G / H$ бинарную операцию, положив $(g_1H) \cdot (g_2H) := (g_1g_2)H$
для любых $g_1, g_2 \in G$.

Как это понимать? Мы хотим перемножить два смежных класса и 
получить в результате третий смежный класс. Для этого мы берем
какой-нибудь элемент $g_1$ из первого смежного класса,
элемент $g_2$ из второго смежного класса и объявляем,
что результатом перемножения наших двух смежных классов
будет смежный класс элемента $g_1g_2$. 
Однако тут возникает потенциальная проблема:
а вдруг при другом выборе элементов $g_1$ и $g_2$ из тех же смежных классов
смежный класс элемента $g_1g_2$ окажется другим?
Оказывается, в нашей ситуации такое невозможно, 
что доказывается так называемой \textit{проверкой корректности}.

Корректность: пусть элементы $g_1', g_2' \in G$ таковы, что 
$g_1'H = g_1H$ и $g_2'H = g_2H$ (то есть $g_1'$ и $g_2'$ --- другие
представители наших исходных смежных классов $g_1H$ и $g_2H$ соответственно).
Тогда $g_1' = g_1h_1$ и $g_2' = g_2h_2$ для некоторых $h_1, h_2 \in H$.
Следовательно,
\[
    (g_1'H) \cdot (g_2'H) = (g_1'g_2')H = (g_1h_1g_2h_2)H = 
    (g_1g_2\underbrace{g_2^{-1}h_1g_2}_{\in H}h_2)H \subseteq (g_1g_2)H \To 
    (g_1'g_2')H = (g_1g_2)H
\]

Итак, на множестве $G / H$ корректно определена бинарная операция.
Теперь легко проверить, что $(G / H, \cdot)$ является группой:
\vspace{-0.2cm}
\begin{itemize}
    \item Ассоциативность:
    \[
        ((aH)(bH))(cH) = ((ab)H)(cH) = ((ab)c)H = (a(bc))H = (aH)((bc)H) = 
        (aH)((bH)(cH));
    \]
    \item Нейтральный элемент --- это $eH$:
    \[
        (eH)(aH) = (ea)H = aH = (ae)H = (aH)(eH);
    \]
    \item Обратный к $gH$ элемент --- это $g^{-1}H$:
    \[
        (g^{-1}H)(gH) = (g^{-1}g)H = H = (gg^{-1})H = (gH)(g^{-1}H).
    \]
\end{itemize}

\begin{definition}
    Группа $(G / H, \cdot)$ называется \textit{факторгруппой} группы $G$
    по нормальной подгруппе $H$.
\end{definition}

\begin{example}
    Пусть $G = (\ZZ, +)$ и $H = n\ZZ$ для некоторого $n \in \NN$.
    Тогда $G / H = (\ZZ_n, +)$ --- знакомая нам группа вычетов.
\end{example}

Подчеркнем, что группу $(\ZZ_n, +)$ довольно затруднительно определить в обход
конструкции факторгруппы.

%%%%%%%%%%%%%%%%%%%%%%%%%%%%%%%%%%%%%%%%%%%%
%%%%%%%%%%%%%%%%%%%%%%%%%%%%%%%%%%%%%%%%%%%%
%%%%%%%%%%%%%%%%%%%%%%%%%%%%%%%%%%%%%%%%%%%% 6
\newpage
\mysection

Пусть $G, F$ --- две группы.

\subsection{Гомоморфизмы групп.}
\begin{definition}
    Отображение $\varphi: G \to F$ называется \textit{гомоморфизмом}, если 
    $\varphi(ab) = \varphi(a) \cdot \varphi(b)$ для любых $a, b \in G$.
\end{definition}


\subsection{Простейшие свойства гомоморфизмов.}

 \hspace{0.3cm} Пусть $\varphi: G \to F$ --- гомоморфизм групп, 
 тогда выполнены следующие простейшие свойства:
 \vspace{-0.3cm}
 \begin{enumerate}[label=\arabic*)]
     \item Если $e_G \in G$ и $e_F \in F$ --- нейтральные элементы,
     то $\varphi(e_G) = e_F$. Иными словами,
     при гомоморфизме групп нейтральный элемент переходит в нейтральный.
     Действительно, имеем
     \[
         \varphi(e_G) = \varphi(e_G \cdot e_G) = \varphi(e_G) \cdot \varphi(e_G).
     \]
     Умножая на $\varphi(e_G)^{-1}$ (слева или справа --- без разницы), 
     получаем $e_F = \varphi(e_G)$.
     \item $\varphi(a^{-1}) = \left( \varphi(a) \right)^{-1}$ для всякого $a \in G$.
     Иными словами, при гомоморфизме обратный элемент переходит в обратный.
     Действительно, имеем 
     $\varphi(a^{-1}) \cdot \varphi(a) = \varphi(a^{-1} \cdot a) = \varphi(e_G) = e_F$
     и аналогично $\varphi(a) \cdot \varphi(a^{-1}) = e_F$, откуда и следует требуемое.
 \end{enumerate}

\subsection{Изоморфизмы групп.}

\begin{definition}
    Гомоморфизм $\varphi: G \to F$ называется \textit{изоморфизмом}, если $\varphi$ --- биекция.
\end{definition}

\begin{comment}
    Если $\varphi: G \to F$ --- изоморфизм, то обратное отображение $\varphi^{-1}: F \to G$ --- 
    тоже изоморфизм. Обратное отображение сохраняет биективность, остается проверить на гомоморфизм:
    пусть $\varphi(a) = a'$ и $\varphi(b) = b'$, тогда получаем
    \[
       \varphi^{-1}(a'b') = \varphi^{-1}(\varphi(a) \cdot \varphi(b)) = 
       \underbrace{\varphi^{-1}(\varphi}_{Id}(ab)) = ab = \varphi^{-1}(a') \cdot \varphi^{-1}(b')
    \]
    получили, что $\varphi^{-1}$ --- это биекция и гомоморфизм $\To$ изоморфизм.
\end{comment}

\begin{definition}
    Группы $G, F$ называются \textit{изоморфными}, если существует изоморфизм
    $\varphi: G \to F$.
\end{definition}
Обозначение: $G \simeq F, \ G \cong F,\ G \MapsTo F$.

Можно показать, что отношение <<$G$ изоморфна $F$>>
на множестве всех групп является отношением эквивалентности,
и тогда все группы разбиваются на классы изоморфизма таким образом,
что внутри одного класса все группы изоморфны между собой.
Изоморфные группы с алгебраической точки зрения рассматриваются как
<<одинаковые>>, и в этом основная ценность самого понятия изоморфности.

\begin{example}
    Отображение взятия экспоненты $\varphi: \RR \to \RR_{>0}, a \mapsto e^a$,
    является изоморфизмом между группами $(\RR, +)$ и $(\RR_{>0}, \times)$.
    Обратный изоморфизм дается отображением логарифмирования $a \mapsto \ln a$.
\end{example}

\subsection{Ядро и образ гомоморфизма групп, их свойства.}

Пусть $\varphi: G \to F$ --- гомоморфизм групп.
\begin{definition}
    \textit{Ядро} гомоморфизм $\varphi$ --- это множество 
    $\ker \varphi := \left\{ g \in G \mid \varphi(g) = e_F \right\} \subseteq G$. \par
    \textit{Образ} гомоморфизма $\varphi$ --- это множество 
    $\Im \varphi := \varphi(G) \subseteq F$.
\end{definition}

\begin{properties}
    \begin{enumerate}
        \item $\Ker \varphi$ --- подгруппа в $G$, $\Im \varphi$ --- подгруппа $F$.
        Проверка:
        \begin{itemize}
            \item Принадлежность нейтрального элемента: \\ 
            из простейших свойств гомоморфизма получаем,
            что $\varphi(e_G) = e_F \To e_G \in \Ker \varphi$ и $e_F \in \Im \varphi$.
            \item Замкнутость множества относительно бинарной операции: \\ 
            Ядро: пусть $a, b \in \Ker \varphi$, тогда 
            $\varphi(ab) = \varphi(a) \cdot \varphi(b) = e_F \cdot e_F = e_F
            \To ab \in \Ker \varphi$. \\
            Образ: пусть $a', b' \in \Im \varphi$,
            где $\varphi(a) = a'$ и $\varphi(b) = b'$, 
            тогда $\varphi(ab) = \varphi(a) \cdot \varphi(b) = a'b' \To a'b' \in \Im \varphi$.
            \item Принадлежность обратного элемента: \\
            Ядро: пусть $a \in \Ker \varphi$, тогда
            $e_F = \varphi(e_G) = \varphi(a \cdot a^{-1}) = \varphi(a) \cdot \varphi(a^{-1}) =
            e_F \cdot \varphi(a^{-1}) = \varphi(a^{-1}) \To a^{-1} \in \Ker \varphi$.\\
            Образ: пусть $y \in \Im \varphi$, где $\varphi(x) = y$, тогда 
            $y^{-1} = \left( \varphi(x) \right)^{-1} = \varphi(x^{-1})
            \To y^{-1} \in \Im \varphi$, 
            в последнем переходе воспользовались простейшим свойством гомоморфизма.
        \end{itemize}
        \item $\varphi$ инъективен тогда и только тогда, когда $\Ker \varphi = \{e\}$.

        \begin{proof}
            От противного: предположим, что $f(a) = f(b)$, где $a \ne b$,
            тогда домножим справа на $f(b^{-1})$ и
            получим $f(a) \cdot f(b^{-1}) = e_F \Eq f(ab^{-1}) = e_F \To
            ab^{-1} = e_G \To a = b$.
        \end{proof}
        \item $\varphi$ --- изоморфизм тогда и только тогда, 
        когда $\Ker \varphi = \{e\}$ и $\Im \varphi = F$.
        
        \begin{proof}
            Из условия известно, что $\varphi$ гомоморфизм, получаем, что
            в данном случае изоморфизм равносилен биективности. 
            
            Известно, что отображение
            называется биекцией, когда оно одновременно инъективно и суръективно.
            Условие $\Im \varphi = F$ это прямое следствие из определения суръективности,
            а эквивалентность условия $\Ker \varphi = \{e\}$ и инъективности 
            следует из предыдущего свойства.
        \end{proof}
    \end{enumerate}
\end{properties}


\begin{lemma}
    $\Ker \varphi$ --- нормальная подгруппа в $G$.
\end{lemma}
\begin{proof}
    Покажем, что $g(\Ker \varphi)g^{-1} \subseteq \Ker \varphi$
    для всякого $g \in G$. Действительно, для каждого
    $x \in \Ker \varphi$ имеем 
    $\varphi(gxg^{-1}) = \varphi(g)\cdot \varphi(x)\cdot \varphi(g^{-1}) =
    \varphi(g) \cdot e_F \cdot \varphi(g^{-1}) =
    \varphi(g) \cdot \left( \varphi(g) \right)^{-1} = e_F$, откуда
    $gxg^{-1} \in \Ker \varphi$, что и требовалось.
\end{proof}

Из леммы следует, что определена факторгруппа $G / \Ker \varphi$.
%%%%%%%%%%%%%%%%%%%%%%%%%%%%%%%%%%%%%%%%%%%%
%%%%%%%%%%%%%%%%%%%%%%%%%%%%%%%%%%%%%%%%%%%%
%%%%%%%%%%%%%%%%%%%%%%%%%%%%%%%%%%%%%%%%%%%% 7
\newpage
\mysection
\subsection{Теорема о гомоморфизме для групп.}

\begin{theorem}
    $G / \Ker \varphi \simeq \Im \varphi$.
\end{theorem}

\begin{proof}
    Определим отображение $\psi: G / \Ker \varphi \to \Im \varphi$, положив
    $\psi(g\Ker \varphi) := \varphi(g)$ для всех $g \in G$.

    1) Корректность: если $g\Ker \varphi = g'\Ker \varphi$, то
    $g' = gh$ для некоторого $h \in \Ker \varphi$, и тогда
    \[
        \psi(g'\Ker \varphi) = \varphi(g') = \varphi(gh) = \varphi(g) \cdot
        \varphi(h) = \varphi(g) \cdot e = \varphi(g) = \psi(g\Ker\varphi).
    \]

    2) Покажем, что $\varphi$ --- гомоморфизм. Имеем
    \[
        \psi((g_1\Ker \varphi)(g_2\Ker\varphi)) = \psi((g_1g_2)\Ker\varphi) =
        \varphi(g_1g_2) = \varphi(g_1) \cdot \varphi(g_2) = 
        \psi(g_1\Ker\varphi) \cdot \psi(g_2\Ker\varphi).
    \]
    
    3) Отображение $\psi$ сюръективно по определению.

    4) Проверим, что $\psi$ инъективно.
    Пусть $\psi(g_1\Ker\varphi) = \psi(g_2\Ker\varphi)$ для некоторых
    $g_1, g_2 \in G$, тогда
    \[
        \varphi(g_1) = \varphi(g_2) \To e = \varphi(g_1)^{-1} \varphi(g_2) =
        \varphi(g_1^{-1}g_2) \To g_1^{-1}g_2 \in \Ker \varphi \To g_1 \Ker \varphi = 
        g_2\Ker\varphi.
    \]

    Таким образом, мы показали, что $\psi$ является изоморфизмом.
\end{proof}
\vspace{0.3cm}
\begin{example}[ы]
    \setlength{\parindent}{0in} 
    1) Пусть $G = (\RR, +)$, $H = (\ZZ, +)$, что представляет
    собой факторгруппа $G / H$?

    Положим $F = (\CC \setminus \{0\}, \times)$ и рассмотрим отображение
    $\varphi: G \to F, a \mapsto e^{2\pi i a} = \cos(2\pi a) + i\sin(2\pi a)$.
    Легко видеть, что $\varphi$ --- гомоморфизм. Тогда $\Im \varphi$ есть просто
    единичная окружность \\ $S^{1} := \{z \in \CC \mid |z| = 1\}$
    (с операцией умножения, она же сложение аргументов = углов).
    С другой стороны, легко видеть, что $\Ker \varphi = H$,
    и по теореме о гомоморфизме мы окончательно получаем $G / H \simeq S^1$.
    
    2)Для произвольной группы $G$ рассмотрим тождественное отображение 
    $id: G \to G, g \mapsto g$, по очевидным причинам оно является гомоморфизмом группы $G$
    в себя. Тогда $\Ker id = \{e\}, \Im id = G$, и по теореме о гомоморфизме мы
    получаем $G / \{e\} \simeq G$.
    
    3)Снова возьмем произвольную группу $G$ и рассмотрим отображение
    $\varphi: G \to \{e\}, g \mapsto e$, оно также является гомоморфизмом.
    Тогда $\Ker id = \{e\}, \Im id = G$, и по теореме о гомоморфизме мы
    получаем $G / \{e\} \simeq G$.
\end{example}

\vspace{0.4cm}

Отметим, что в примерах $2)$ и $3)$ мы вычислили факторгруппы
произвольной группы по ее обеим несобственным подгруппам.


%%%%%%%%%%%%%%%%%%%%%%%%%%%%%%%%%%%%%%%%%%%%
%%%%%%%%%%%%%%%%%%%%%%%%%%%%%%%%%%%%%%%%%%%%
%%%%%%%%%%%%%%%%%%%%%%%%%%%%%%%%%%%%%%%%%%%% 8
\newpage
\mysection
\subsection{Классификация циклических групп.}


\begin{proposal}
    Пусть $G$ --- циклическая группа. Тогда

    (a) $|G| = \infty \To G \simeq (\ZZ, +)$;

    (б) $|G| = n \To G \simeq (\ZZ_n, +)$.
\end{proposal}

\begin{proof}
    Пусть $G = \left< g \right>$. Рассмотрим отображение 
    $\varphi: \ZZ \to G, k \mapsto g^{k}$.
    Тогда $\varphi(k + l) = g^{k + l} = g^kg^l = \varphi(k)\varphi(l)$,
    поэтому $\varphi$ --- гомоморфизм. Из определения циклической группы следует,
    что $\varphi$ сюръективен, то есть $\Im \varphi = G$. Тогда по теореме о
    гомоморфизме мы получаем $G \simeq \ZZ / \Ker \varphi$.
    Так как $\Ker \varphi$ --- подгруппа в $\ZZ$, то по предложению 
    \ref{pr:1:1} получаем $\Ker \varphi = m\ZZ$ для некоторого 
    $m \geq 0$. 
    
    Если $m = 0$, то 
    $\Ker \varphi = \{0\}$, откуда $G \simeq \ZZ / \{0\} \simeq \ZZ$.
    
    Если $m > 0$, то $G \simeq \ZZ / m\ZZ = \ZZ_m$.
\end{proof}