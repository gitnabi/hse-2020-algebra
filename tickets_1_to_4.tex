%%%%%%%%%%%%%%%%%%%%%%%%%%%%%%%%%%%%%%%%%%%%
%%%%%%%%%%%%%%%%%%%%%%%%%%%%%%%%%%%%%%%%%%%%
%%%%%%%%%%%%%%%%%%%%%%%%%%%%%%%%%%%%%%%%%%%% 1
\mysection
\subsection{Бинарные операции.}
Пусть $M$ --- некоторое множество.
\begin{definition}
  Бинарная операция на множестве $M$ это отображение
  $\circ: M \times M \to M,\ (a, b) \mapsto a \circ b$. 
\end{definition}
Если на $M$ задана бинарная операция, то пару $(M, \circ)$ называют
\textit{множеством с бинарной операцией}.

Классические примеры с бинарной операцией, знакомые со школы, ---
это операция сложение$(+)$ и умножение$(\times)$ на множестве 
$\NN, \ZZ, \QQ, \RR$.


%-------------------------------------------


\subsection{Полугруппы, моноиды и группы.}
Пусть $(M, \circ)$ --- множество с бинарной операцией.

\begin{definition}
  \vspace{-0.2cm}
  \begin{enumerate}[label=\Roman*.]
    \item $(M, \circ)$ называется \textit{группой}, 
    если выполнены следующие три условия(аксиомы):
    \begin{enumerate}[label=(\arabic*)]
      \item $(a \circ b) \circ c = a \circ (b \circ c)
      \ \ \forall a, b, c \in  M$(\textit{ассоциативность});
      \item существует \textit{нейтральный элемент}, то есть
      такой $e \in M$, что $e \circ a = a \circ e = a, \ \forall a \in M$;
      \item для всякого $a \in M$ существует \textit{обратный элемент},
      то есть такой $b \in M$, что $a \circ b = b \circ a = e$.
    \end{enumerate}
    \item Если требуется выполнение только условия (1),
    то $(M, \circ)$ называется \textit{полугруппой}.
    \item Если требуется выполнение только условий (1) и (2),
    то $(M, \circ)$ называется \textit{моноидом}.
  \end{enumerate}
\end{definition}

\begin{example}[ы для {\normalfont II} и {\normalfont III}]
  $(\NN, +)$ --- это полугруппа, но не моноид.
  
  $(\NN \cup \{0\}, +)$ --- моноид, но не группа.
\end{example}
\\ \\
\begin{comment}
  \vspace{-0.3cm}
  \begin{enumerate}
    \item Ассоциативность довольно редкое свойство. \\
     Примеры неассоциативных бинарных операций:
     $M = \ZZ, a \circ b := a - b$ или 
     $M = \NN, a \circ b := a^b$.
    \item Нейтральный элемент в моноиде (и группе) единствен:
    если $e_1, e_2 \in M$ --- два нейтральных элемента, то
    $e_1 = e_1 \circ e_2 = e_2$.
    \item Обратный элемент в группе единствен:
    если $b_1, b_2$ --- два обратных к $a$ элемента, то
    \[
     b_1 = b_1 \circ e = b_1 \circ (a \circ b_2) =
     (b_1 \circ a) \circ b_2 = e \circ b_2 = b_2.
    \]
    Ввиду единственности обратный к $a$ элемент обозначается 
    символом $a^{-1}$.
    \item $(a \circ b)^{-1} = b^{-1} \circ a^{-1}$:
    \[
      (a \circ b) \circ (b^{-1} \circ a^{-1}) = 
      a \circ (b \circ b^{-1}) \circ a^{-1} = 
      a \circ e \circ a^{-1} = a \circ a^{-1} = e.
    \]
  \end{enumerate}
\end{comment}

\textbf{\textsf{Соглашение}}:
вместо $(G, \circ)$ будем писать $G$, 
вместо $a \circ b$ будем писать $ab$ и 
операцию $\circ$ будем называть умножением.
%-------------------------------------------

\subsection{Коммутативные группы.}
\vspace{0.2cm}
\begin{definition}
  Группа $G$ называется \textit{коммутативной}
  (или \textit{абелевой}), если $ab = ba$ для всех $a, b \in G$.
\end{definition}

При работе с абстрактными группами для обозначения 
групповой операции, нейтрального и обратного 
элементов принято использовать 
\textit{мультипликативную запись}: $ab,\ e,\ a^{-1}$.
Однако в теории абелевых групп употребляется 
\textit{аддитивная запись}: $a + b,\ 0,\ -a$
(при этом сама операция называется сложением).
 

%-------------------------------------------
\vspace{-0.2cm}
\subsection{Примеры групп.}
1) Числовые аддитивные группы:
  $(\ZZ, +), \ (\QQ, +), \ (\RR, +), \ (\CC, +), \ (\ZZ_n, +)$.

2) Числовые мультипликативные группы:
   $(\QQ \setminus \{0\}, \times), \ (\RR \setminus \{0\}, \times),
   (\CC \setminus \{0\}, \times),
   (\ZZ_p \setminus \{\overline{0}\}, \times), p$ --- простое.

3) Группы матриц (с операцией умножения):

$\GL_n(\RR) = \{A \in \M_n(\RR) \mid \det A \ne 0\}$ --- 
полная линейная группа;

$\SL_n(\RR) = \{A \in \M_n(\RR) \mid \det A = 1\}$ --- 
специальная линейная группа.

4) Группы перестановок (с операцией композиции):

симметрическая группа $S_n$ --- все перестановки длины $n$,
$|S_n| = n!$.

знакопеременная группа $A_n$ --- все четные перестановки длины $n$,
$|A_n| = n! /2$.

%-------------------------------------------
\vspace{-0.2cm}
\subsection{Порядок группы.}
\begin{definition}
  \textit{Порядком} группы $G$ называется число элементов в ней.
\end{definition}

Группа называется \textit{конечной}, если ее порядок конечен,
и \textit{бесконечной} иначе.

Обозначение: $|G|$.
%-------------------------------------------
\vspace{-0.2cm}
\subsection{Подгруппы.}
\begin{definition}
  Подмножество $H$ группы $G$ называется \textit{подгруппой},
  если
  \vspace{-0.2cm}
  \begin{enumerate}[label=\arabic*)]
    \item $e \in H$;
    \item $a, b \in H \To ab \in H$;
    \item $a \in H \To a^{-1} \in H$;
  \end{enumerate}
\end{definition}

В каждой группе есть \textit{несобственные}
подгруппы $H = \{e\}$ и $H = G$.

Остальные подгруппы называются \textit{собственными}.
%-------------------------------------------
\vspace{-0.2cm}
\subsection{Описание всех подгрупп в группе \texorpdfstring{$(\ZZ, +)$.}{Lg}}
\begin{proposal}\label{pr:1:1}
  Всякая подгруппа в $(\ZZ, +)$ имеет вид $k\ZZ$
  для некоторого $k \geqslant 0$. 
\end{proposal}

\begin{proof}
  Пусть $H \subseteq \ZZ$ --- некоторая подгруппа.
  Если $H = \{0\}$, то $H = 0 \cdot \ZZ$.
  Далее считаем $H \ne \{0\}$.
  По определению подгруппы для всякого $x \in H$ имеем
  $-x \in H$, поэтому множество $H \cap \NN$ не пусто,
  и мы положим $k = \min(H \cap \NN)$.
  Тогда опять же по определению подгруппы получаем
  $k\ZZ \subseteq H$.
  Пусть теперь $a \in H$ --- произвольный элемент.
  Поделим его на $k$ с остатком: $a = qk + r, \ 0 \leqslant r < k$.
  Снова воспользовавшись определением подгруппы, мы получаем
  $r = a - qk \in H$, откуда в силу минимальности $k$
  вытекает $r = 0$ и $a \in k\ZZ$. Значит, $k\ZZ = H$.
\end{proof}


%%%%%%%%%%%%%%%%%%%%%%%%%%%%%%%%%%%%%%%%%%%%
%%%%%%%%%%%%%%%%%%%%%%%%%%%%%%%%%%%%%%%%%%%%
%%%%%%%%%%%%%%%%%%%%%%%%%%%%%%%%%%%%%%%%%%%% 2
\newpage
\mysection
\subsection{Подгруппы.}
Пусть $G$ --- группа, $g \in G$ и $n \in \ZZ$.
Определим $n$-ю степень $g^n$ следующим образом:
\[
  g^n :=
  \begin{cases}
    \underbrace{g \cdot \ldots \cdot g}_n, & \text{если } n > 0; \\
    e, & \text{если } n = 0; \vspace{0.3cm} \\
    \underbrace{g^{-1} \cdot \ldots \cdot g^{-1}}_{|n|},
    & \text{если } n < 0.
  \end{cases}
\]

\begin{comment}
  Если $G$ --- абелева группа с операцией сложения,
  то в аддитивной записи определенная выше \\ <<$n$-я степень>>
  элемента $g \in G$ будет не чем иным, как $ng$
  (то есть кратным элемента $g$).
\end{comment}
\vspace{0.4cm}

\begin{properties}
  \begin{enumerate}[label = \arabic*)]
    \item $g^ng^m = g^{n + m}$:
    \[
      g^ng^m = \underbrace{g \cdot \ldots \cdot g}_n \cdot
      \underbrace{g \cdot \ldots \cdot g}_m =
      \underbrace{g \cdot \ldots \cdot g}_{m + n} = g^{n + m}; 
    \]
    \item $(g^n)^{-1} = g^{-n}$:
    \[
      g^n \cdot g^{-n} = g^{n - n} = g^0 = e;
    \]
    \item $(g^m)^n = g^{mn}$:
    \[
      (g^m)^n = (\underbrace{g \cdot \ldots \cdot g}_m)^n= 
      \underbrace{
      \underbrace{g \cdot \ldots \cdot g}_m \cdot \ldots \cdot 
      \underbrace{g \cdot \ldots \cdot g}_m}_n = g^{mn}.
    \]
  \end{enumerate}
\end{properties}

Для каждого $g \in G$ положим 
$\left<g\right> := \{g^n \mid n \in \ZZ \}$.
В силу упомянутых выше свойств $\left<g\right>$
является подгруппой в $G$.

%-------------------------------------------
\vspace{0.4cm}

\subsection{Циклические подгруппы.}

\begin{definition}
  Подгруппа $\left<g\right>$ называется
  \textit{циклической подгруппой} в $G$, 
  порождаемой элементом $g$.
  При этом $g$ называется \textit{образующим} или 
  \textit{порождающим} элементом для $\left<g\right>$.
\end{definition}

\begin{example}
  $2\ZZ \in \ZZ$ --- циклическая подгруппа,
  $2\ZZ = \left<2\right> = \left<-2\right>$.
\end{example}

\vspace{0.3cm}

\begin{comment}
  Циклическая подгруппа $\left<g\right>$ всегда коммутативна.
\end{comment}

%-------------------------------------------
\newpage
\subsection{Циклические группы.}
\vspace{0.3cm}

\begin{definition}
    Группа $G$ называется \textit{циклической},
    если существует такое $g \in G$, что $G = \left<g\right>$.
\end{definition}

\begin{example}
  Группы $(\ZZ, +), (\ZZ_n, +)$ при $n \geqslant 1$ 
  являются циклическими.
\end{example}

\begin{comment}
  Если $G$ --- циклическая группа,
  то $G$ коммутативна и не более чем счётна.
\end{comment}

%-------------------------------------------

\subsection{Порядок элемента.}
Пусть $G$ --- группа и $g \in G$.
Рассмотрим множество $M(g) = \{n \in \NN \mid g^n = e\}$.

\begin{definition}
  \textit{Порядок} элемента $g$ --- это величина
  \[
    \ord(g) :=
    \begin{cases}
      \min M(g), & \text{если } M(g) \ne \varnothing; \\
      \infty, & \text{если } M(g) = \varnothing.
    \end{cases}
  \]
\end{definition}

\begin{comment}
  $\ord(g) = 1 \ \Eq \ g = e$.
\end{comment}

%-------------------------------------------

\subsection{Связь между порядком элемента и порядком
            порождаемой им циклической подгруппы.}
\vspace{0.3cm}

\begin{proposal}\label{pr:2:5}
  $\ord(g) = \left|\left<g\right>\right|$.
\end{proposal}

\begin{proof}
  Пусть $k, s \in \ZZ$, такие что $g^k = g^s$.
  Тогда домножая обе части равенства на $g^{-s}$, получаем, что
  \[
    g^k = g^s \Eq g^{k - s} = e. \tag{$*$} \label{star:first}
  \]
  
  Далее рассмотрим два случая.
  \vspace{-0.2cm}
  \begin{enumerate}[label=\arabic*)]
    \item Если $\ord(g) = \infty$,
    то нет таких двух различных чисел $k, s \in \ZZ$, что $g^{k - s} = e$,
    а значит, в силу \eqref{star:first} все элементы в подгруппе различны 
    $\ \To \ \left|\left<g\right>\right| = \infty$.
    \item Если $\ord(g) = m < \infty$,
    то элементы $e = g^0, g = g^1, g^2, \ldots, g^{m-1}$ попарно различны
    в силу \eqref{star:first}.
    
    Покажем, что эти элементы исчерпывают всю группу $\left<g\right>$.
    Возьмем произвольное $n \in \ZZ$ и разделим его на $m$ с остатком:
    $n = qm + r, \ 0 \leqslant r < m$. 
    Тогда $g^n = g^{qm} \cdot g^r = (g^m)^q \cdot g^r = e^q \cdot g^r = e \cdot g^r = g^r$.
    
    Получили, что всякий элемент из $\left<g\right>$ принадлежит 
    $\{e, g, g^2, \dots, g^{m - 1}\} \ \To \
    \left<g\right> = \{e, g, g^2, \dots, g^{m - 1}\}$, значит,
    $\left|\left<g\right>\right| = m$.
  \end{enumerate}
\end{proof}

%%%%%%%%%%%%%%%%%%%%%%%%%%%%%%%%%%%%%%%%%%%%
%%%%%%%%%%%%%%%%%%%%%%%%%%%%%%%%%%%%%%%%%%%%
%%%%%%%%%%%%%%%%%%%%%%%%%%%%%%%%%%%%%%%%%%%% 3
\newpage
\mysection
\subsection{Смежные классы.}

Пусть теперь $G$ --- группа, $H \subseteq G$ --- подгруппа.
Определим на $G$ отношение $L_H$ следующим образом:
$(a, b) \in L_H \Eq a^{-1}b \in H$.

\begin{proposal}\label{pr:first}
  $L_H$ --- отношение эквивалентности.
\end{proposal}

\begin{proof}
  Проверим все необходимые свойства, они вытекают аккурат
  из определения подгруппы.
  \vspace{-0.2cm}
  \begin{enumerate}
    \item Рефлексивность: $a^{-1}a = e \in H$.
    \item Симметричность: $a^{-1}b \in H \To b^{-1}a = (a^{-1}b)^{-1} \in H$.
    \item Транзитивность:
    $a^{-1}b \in H, b^{-1}c \in H \To a^{-1}c = (a^{-1}b)(b^{-1}c) \in H$.
  \end{enumerate}
  \vspace{-0.2cm}
\end{proof}

Теперь заметим, что $a^{-1}b \in H \Eq b \in aH$,
поэтому класс эквивалентности элемента $a \in G$ для отношения
$L_H$ совпадает с множеством $aH$.
\vspace{0.4cm}
\begin{definition}
  \hspace{-0.5cm}
  Множество $aH := \{ah \mid h \in H\}$ называется
  \textit{левым смежным классом} элемента $a \in G$ 
  по подгруппе $H$.
\end{definition}

Из предложения и общих фактов об отношениях эквивалентности вытекает,
что группа $G$ разбивается в объединение попарно непересекающихся левых
смежных классов по подгруппе $H$.

Следующая лемма показывает, что в случае конечной подгруппы $H$
все левые смежные классы содержат одинаковое число элементов, равное $|H|$.

\begin{lemma}\label{le:first}
  Если $|H| < \infty$, то $|aH| = |H|$ для всех $a \in G$.
\end{lemma}

\begin{proof}
  Из определения следует, что $|aH| \leqslant |H|$.
  Если $ah_1 = ah_2$ для каких-то $h_1, h_2 \in H$, то,
  умножая на $a^{-1}$ слева, получаем $h_1 = h_2$, откуда $|aH| = |H|$.  
\end{proof}


%-------------------------------------------
\subsection{Индекс подгруппы.}
\vspace{0.5cm}
\begin{definition}
  \textit{Индекс} подгруппы $H$ в группе $G$ ---
  это число левых смежных классов $G$ по $H$.
\end{definition}

Обозначение: $[G : H]$.

Отметим, что индекс подгруппы ---
это либо натуральное число, либо бесконечность.

%-------------------------------------------
\newpage
\subsection{Теорема Лагранжа.}
\vspace{0.2cm}
\begin{theorem}
  Пусть $G$ --- конечная группа, $H \subseteq G$ --- подгруппа,
  тогда $|G| = |H| \cdot [G : H]$.  
\end{theorem}

\begin{proof}
  Следует из предложения и леммы: группа $G$ разбивается 
  в объединение попарно непересекающихся
  левых смежных классов в количестве $[G : H]$ штук,
   а каждый класс содержит ровно $|H|$
  элементов. 
\end{proof}

%%%%%%%%%%%%%%%%%%%%%%%%%%%%%%%%%%%%%%%%%%%%
%%%%%%%%%%%%%%%%%%%%%%%%%%%%%%%%%%%%%%%%%%%%
%%%%%%%%%%%%%%%%%%%%%%%%%%%%%%%%%%%%%%%%%%%% 4
\newpage
\mysection
\subsection{Пять следствий из теоремы Лагранжа.}

\begin{corollary}[ 1]
  Пусть $G$ --- конечная группа и $H \subseteq G$ --- подгруппа. 
  Тогда $|H|$ делит $|G|$.
\end{corollary}

\begin{corollary}[ 2]
  Пусть $G$ --- конечная группа и $g \in G$. Тогда $\ord(g)$ делит $|G|$.
\end{corollary}

\begin{proof}
  Это вытекает из следствия 1 и предложения \ref{pr:2:5}.
\end{proof}

\begin{corollary}[ 3]
  Пусть $G$ --- конечная группа и $g \in G$. Тогда $g^{|G|} = e$.
\end{corollary}

\begin{proof}
  Согласно следствию 2, мы имеем $|G| = \ord(g) \cdot s$,
  откуда $g^{|G|} = \left(g^{\ord(g)}\right)^{s} = e^{s} = e.$
\end{proof}

\begin{corollary}[ 4 \textnormal{(Малая теорема Ферма)}]
  Пусть $\bar a$ --- ненулевой вычет по простому модулю $p$. Тогда
  \[
    \bar a^{p - 1} = \bar 1.
  \]
\end{corollary}

\vspace{-0.4cm}

\begin{proof}
  Вытекает из следствия 3, примененного к группе 
  ($\ZZ_p \setminus \{\bar0\}, \times$).
\end{proof}

\begin{corollary}[ 5]
  Пусть $G$ --- группа и $|G|$ --- простое число. 
  Тогда $G$ --- циклическая группа, 
  порождаемая любым своим неединичным элементном. 
\end{corollary}

\begin{proof}
  Пусть $g \in G$ --- произвольный неединичный элемент.
  Тогда циклическая подгруппа $\left< g \right>$
  содержит более одного элемента и $\left| \left< g \right> \right|$
  делит $\left| G \right|$ по следствию 1.
  Так как $\left| G \right|$ --- простое число, то
  последнее возможно только при $\left| \left< g \right> \right| 
  = \left| G \right|$, откуда $G = \left< g \right>$.
\end{proof}

