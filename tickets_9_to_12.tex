%%%%%%%%%%%%%%%%%%%%%%%%%%%%%%%%%%%%%%%%%%%%
%%%%%%%%%%%%%%%%%%%%%%%%%%%%%%%%%%%%%%%%%%%%
%%%%%%%%%%%%%%%%%%%%%%%%%%%%%%%%%%%%%%%%%%%% 9
\mysection
\subsection{Прямое произведение групп.}

Пусть $G_1 , \ldots , G_m$ --- группы.

\begin{definition}\label{de:9:1}
    \textit{Прямое произведение} групп $G_1, \ldots , G_m$ --- это множество
    $G_1 \times \ldots \times G_m$ с бинарной операцией
    $(g_1, \ldots , g_m)(g_1', \ldots , g_m') := (g_1g_1', \ldots , g_mg_m')$.
\end{definition}

Проверим, что $G_1 \times \ldots \times G_m$ --- действительно группа:

\begin{enumerate}
    \item Ассоциативность:
    \[
        ((g_1, \ldots , g_m)(g_1', \ldots , g'm))(g_1'',\ldots ,g_m'') =
        ((g_1g_1')g_1'', \ldots , (g_mg_m')g_m'') = 
    \]
    \[
        = (g_1(g_1'g_1''), \ldots , g_m(g_m'g_m'')) = 
        (g_1, \ldots , g_m)((g_1', \ldots , g_m')(g_1'', \ldots , g_m''));
    \]
    \item Нейтральный элемент --- $e_{G_1}, \ldots , e_{G_m}$:
    \[
        (e_{G_1}, \ldots , e_{G_m})(g_1, \ldots , g_m) = 
        (e_{G_1}g_1, \ldots , e_{G_m}g_m) = (g_1, \ldots , g_m) = 
    \]
    \[
        = (g_1e_{G_1}, \ldots , g_me_{G_m}) = 
        (g_1, \ldots , g_m)(e_{G_1}, \ldots , e_{G_m});
    \]
    \item Обратный к $(g_1, \ldots , g_m)$ элемент --- этo
    $(g_1^{-1}, \ldots , g_m^{-1})$:
    \[
        (g_1, \ldots , g_m)(g_1^{-1}, \ldots , g_m^{-1}) = 
        (g_1g_1^{-1}, \ldots , g_mg_m^{-1}) = (e_{G_1}, \ldots , e_{G_m}) =
    \]
    \[
        = (g_1^{-1}g_1, \ldots , g_m^{-1}g_m) = 
        (g_1^{-1}, \ldots , g_m^{-1})(g_1, \ldots , g_m)
    \]
\end{enumerate}

Как видно, и тут все необходимые свойства вытекают из того, что
$G_1, \ldots , G_m$ --- группы.

\begin{example}
    1) $\underbrace{\RR \times \ldots \times \RR}_{n} = \RR^n$ ---
    знакомый нам из курса линейной алгебры объект.

    2) $\underbrace{\ZZ \times \ldots \times \ZZ}_{n} = \ZZ^n$ ---
    это подгруппа в $\RR^n$, состоящая из всех векторов с целочисленными
    координатами.

    Группа $\ZZ^n$ называется \textit{решеткой} ранга $n$.
\end{example}

\begin{comment}
    1) Группа $G_1 \times \ldots \times_m$ абелева тогда и только тогда,
    когда все группы $G_1, \ldots , G_m$ абелевы.

    2) Если группы $G_1, \ldots , G_m$ конечны, то 
    $|G_1 \times \ldots \times G_m| = |G_1| \cdot \ldots \cdot |G_m|$.
\end{comment}

Обратим внимание, что для каждого $i = 1, \ldots , m$ естественный
гомоморфизм
\[
    G_i \to G, 
    g \mapsto (e_{G_1}, \ldots , e_{G_{i - 1}}, g, e_{G_{i + 1}}, \ldots, e_{G_m}),
\]
является вложением, ввиду чего $G_i$ обычно отождествляется с его образом и
рассматривается как подгруппа 
$\left\{ (e_{G_1}, \ldots ,  e_{G_{i - 1}}, g, e_{G_{i + 1}}, \ldots, e_{G_m}))
\mid g \in G_i \right\}$ группы $G_1 \times \ldots \times G_m$.

Пусть $G$ --- некоторая группа и $H_1, \ldots , H_m$ --- ее подгруппы.

\begin{definition}\label{de:9:2}
    Говорят, что $G$ \textit{разлагается в прямое произведение} своих
    подгрупп $H_1, \ldots , H_m$, если отображение
    $H_1 \times \ldots \times H_m \to G, \ 
    (h_1, \ldots , h_m) \mapsto h_1 \cdot \ldots h_m$, является изоморфизмом.
\end{definition}

В этой ситуации обычно допускают вольность в обозначениях и пишут
$G = H_1 \times \ldots \times H_m$, хотя формально $G$ и
$H_1 \times \ldots \times H_m$ --- это разные группы.
Как говорят, $G$ и $H_1 \times \ldots  \times H_m$ отождествляются
по указанному выше изоморфизму между ними.

Возвращаясь к ситуации определения \ref{de:9:1} и отождествляя
каждую группу $G_i$ с подгруппой в $G_1 \times \ldots \times G_m$,
как выше, мы теперь можем сказать, что группа $G_1 \times \ldots \times G_m$
является прямым произведением своих подгрупп $G_1, \ldots , G_m$.

Конструкцию прямого произведения в смысле определения
\ref{de:9:1} иногда называют внешним прямым произведением,
а в смысле определения \ref{de:9:2} --- 
внутренним прямым произведением.
Однако ввиду сделанных выше отождествлений на практике 
зачастую не делается различий между этими двумя понятиями.

Далее мы будем работать только с абелевыми группами и в соответствии
с этим будем пользоваться аддитивной записью групповой операции.

\subsection{Разложение конечной циклической группы.}

\begin{theorem}
    Пусть числа $n, m, l \in \NN$ такомы, что $n = ml$ и $\NOD(m,l) = 1$.
    Тогда $\ZZ_n \simeq \ZZ_m \times \ZZ_l$.
\end{theorem}

\begin{proof}
    Рассмотрим отображение $\varphi: \ZZ_n \to \ZZ_m \times \ZZ_l, \
    \varphi(a \mod n) := (a \mod m, \ a \mod l)$, и покажем, что 
    оно является изоморфизмом.

    1) Корректность: есть, так как из делимости на $n$ автоматически следует
    делимость на $m$ и $l$.

    2) $\varphi$ --- гомоморфизм:
    \[
        \varphi((a + b) \mod n) = ((a + b) \mod m, (a + b) \mod l) = 
    \]
    \[
        = (a \mod m, a \mod l) + (a \mod m, a \mod l) = 
        \varphi(a \mod n) + \varphi(b \mod n).
    \]
    
    3) Инъективность: если $\varphi(a \mod n) = (0, 0)$, то $a \divby m$ и
    $a \divby l$, откуда $a \divby n$ в силу
    $\NOD(m, l) = 1$, откуда $a \mod n = 0$. Значит, $\Ker \varphi = \{0\}$ и
    $\varphi$ инъективно.

    4) Сюръективность: имеем $|\ZZ_n| = n = m \cdot l = |\ZZ_m \times \ZZ_l|$,
    и тогда требуемое следует из $3)$, поскольку всякое инъективное отображение
    между двумя конечными множествами одной мощности автоматически сюръективно.
\end{proof}

\begin{corollary}[\textnormal{ (\textit{Разложение конечной циклической группы})}]

    Пусть $n \leq 2$ --- натуральное число и
    $n = p_1^{k_1} \cdot \ldots \cdot p_s^{k_s}$ --- его разложение на простые
    множители ($p_i \ne p_j$ при $i \ne j$). 
    Тогда $\ZZ_n \simeq \ZZ_{p_1^{k_1}} \times \ldots \times \ZZ_{p_s^{k_s}}$.
\end{corollary}

\subsection{Теорема о строении конечных абелевых групп.}

\begin{definition}
    Конечная абелева группа $A$ называется \textit{примарной}, если
    $|A| = p^{k}$, где $p$ --- простое и $k \in \NN$.
\end{definition}

Следующая важная теорема дает классификацию всех конечных абелевых групп 
с точностью до изоморфизма. Поскольку этот результат нам не понадобится
в дальнейшем, мы не будем тратить время на его доказательство.

\begin{theorem}
    Пусть $A$ --- конечная абелева группа. 
    Тогда $A \simeq \ZZ_{p_1^{k_1}} \times \ldots \times \ZZ_{p_t^{k_t}}$, где 
    $p_1, \ldots , p_t$ --- простые числа (не обязательно попарно различные!)
    и $k_1, \ldots , k_t \in \NN$. Более того, набор примарных циклических
    множителей $\ZZ_{p_1^{k_1}}, \ldots , \ZZ_{p_t^{k_t}}$ определен
    однозначно с точностью до перестановки.
\end{theorem}

Глобальный смысл этой теоремы заключается в том, что конечные абелевы группы
устроены очень просто и все они получаются конструкцией прямого
произведения из примарных циклических групп
(играющих здесь роль <<кирпичиков>>, из которых все строится). 
Отметим, что частным случаем теоремы  случит следствие 5 из теоремы
Лагранжа, согласно которому (с учетом классификации циклических групп)
вообще любая (не обязательно абелева) конечная группа простого порядка
$p$ изоморфна $\ZZ_p$.

%%%%%%%%%%%%%%%%%%%%%%%%%%%%%%%%%%%%%%%%%%%%
%%%%%%%%%%%%%%%%%%%%%%%%%%%%%%%%%%%%%%%%%%%%
%%%%%%%%%%%%%%%%%%%%%%%%%%%%%%%%%%%%%%%%%%%% 10
\newpage
\mysection
\subsection{Экспонента конечной абелевой группы и критерий цикличности.}

Пусть $A$ --- конечная абелева группа.

\begin{definition}
    \textit{Экспонентой} группы $A$ называется число
    \[
        \exp A := \min\{m \in \NN \mid ma = 0 \text{ для всех } a \in A\}.
    \]
\end{definition}

\begin{comment}
    1) Так как $ma = 0 \Eq m \divby \ord(a)$ для всех $a \in A$ и
    $m \in \ZZ$, то определение экспоненты можно переписать еще в
    таком виде: $\exp A = \NOK\{\ord(a) \mid a \in A\}$.

    2) Так как $|A| \divby \ord(a)$ для всех $a \in A$
    (следствие 2 из теоремы Лагранжа), то $|A|$ ---
    общее кратное множества $\{\ord(a) \mid a \in A\}$,
    а значит, $|A| \divby \exp A$. В частности, $\exp A \leq |A|$.
\end{comment}

В дальнейшем нам понадобится следующий факт\textit{ (Критерий цикличности)},
который показывает, когда в последнем
неравенстве достигается равенство.

\begin{proposal}
    $\exp A = |A| \ \Eq \ $ группа $A$ является циклической.
\end{proposal}

\begin{proof}
    Положим $n = |A|$ и рассмотрим разложение на простые множители:
    $n = p_1^{k_1} \cdot \ldots \cdot p_s^{k_s}$, где $p_i$ ---
    простое и $k_i \in \NN$ для всех $i = 1, \ldots , s, p_i \ne p_j$
    при $i \ne j$.

    ($\Leftarrow$) Если $A = \left< a \right>$, то $\ord(a) = n$,
    откуда в силу неравенства сразу получаем $\exp A = n$.
    
    ($\Rightarrow$) Если $\exp A = n$, то для каждого $i = 1, \ldots , s$
    существует элемент $c_i \in A$, такой что $\ord(c_i) = p_i^{k_i}m_i$, где 
    $m_i \in \NN$. Для каждого $i = 1, \ldots , s$ положим $a_i = m_ic_i$,
    тогда $\ord(a_i) = p_i^{k_i}$. Теперь рассмотрим элемент
    $a = a_1 + \ldots + a_s$ и покажем, что $\ord(a) = n$.
    Пусть $ma = 0$ для некоторого $m \in \NN$, то есть 
    $ma_1 + \ldots + ma_s = 0$.
    При фиксированном $i \in \{1, \ldots , s\}$ умножим обе части последнего
    равенства на $n_i := n / p_i^{k_i}$. Легко видеть, что 
    $mn_ia_j = 0$ при всех $i \ne j$, поэтому в левой части выживет только
    слагаемое $mn_ia_i$, откуда получаем $mn_ia_1 = 0$. Следовательно, 
    $mn_i \divby p_i^{k_i}$, а так как $n_i$ не делится на $p_i$, то
    $m \divby p_i^{k_i}$. В силу произвольности выбора $i$ отсюда
    вытекает, что $m \divby n$. Так как $na = 0$, то мы окончательно получаем
    $\ord(a) = n$. Значит, $A = \left< a \right>$ --- циклическая группа.
\end{proof}


%%%%%%%%%%%%%%%%%%%%%%%%%%%%%%%%%%%%%%%%%%%%
%%%%%%%%%%%%%%%%%%%%%%%%%%%%%%%%%%%%%%%%%%%%
%%%%%%%%%%%%%%%%%%%%%%%%%%%%%%%%%%%%%%%%%%%% 11
\newpage
\mysection
Расскажем об одном весьма элементарном, однако очень важном и
используемом на практике применении конечных циклических групп к
задачам криптографии с открытым ключом. Одним из основных
примеров групп, к которым применяются описанные ниже криптосистемы,
служит мультипликативная группа вычетов $G = (\ZZ_p \setminus \{0\}, \times)$
по простому модулю $p$. Поэтому мы вернемся к мультипликативным обозначениям.

\subsection{Задача дискретного логарифмирования.}
Пусть $G$ --- конечная группа и $g \in G$ --- элемент 
достаточно большого порядка. Для данного элемента $h \in \left< g \right>$
найти такое $k \in \NN$, что $h = g^{k}$.

\subsection{Криптография с открытым ключом.}
Метод шифрования информации с открытым ключом основан на предположении 
о том, что для данных элементов $g$ и $h$ решение задачи дискретного
логарифмирования трудоемко и при подходящих входных данных и текущем уровне
вычислительных мощностей практически не реализуемо. Напротив, возведение
элемента в заданную степень можно произвести достаточно эффективно,
использую, например, метод повторного возведения в квадрат.

Следующий метод позволяет двум участникам переписки у всех на глазах добиться
того, что у них появляется элемент, известный только им двоим.

\subsection{Система Диффи-Хеллмана обмена ключами (1976).}

Всем участникам переписки известны конечная группа $G$ и элемент $g \in G$
достаточно большого порядка. Каждый участник $A$ загадывает
свое натуральное число $a$, которое держит в секрете, и сообщает всем
значение $g^{a}$. После этого каждая пара
участников $A$ и $B$ может составить общий для нее ключ:
$A$ возводит элемент $g^{b}$ в степень $a$, a $B$ возводит
$g^{a}$ в степень $b$. В результате элемент $g^{ab}$ есть только у
$A$ и $B$, и они могут использовать его в качестве ключа для дальнейшей
конфиденциальной переписки.

\subsection{Криптосистема Эль-Гамаля (1985).}

Сначала обсудим основную идею. обсудим основную идею. 
Пусть все так же, как в описании системы Диффи-Хеллмана,
и участники $A$ и $B$ уже сгенерировали свой секретный ключ $g^{ab}$.
В дальнейшем участнику $A$ понадобится также элемент $(g^{ab})^{-1}$,
который, зная $g^b$ и $a$, можно сразу вычислить как $(g^{b})^{|G| - a}$
(вспомним следствие 3 из теоремы Лагранжа). Если теперь участник $B$ хочет
конфиденциально передать участнику $A$ элемент $h \in G$
(кодирующий какое-то важное сообщение), то он вычисляет и сообщает всем
элемент $y = hg^{ab}$. Теперь участник $A$ может восстановить $h$, домножая
$y$ справа на $\left(g^{ab}\right)^{-1}$, 
то есть по формуле $h = y\left(g^{b}\right)^{|G| - a}$.
Заметим, что никто из толпы шпионов, наблюдающих за данной перепиской,
не в состоянии определить элемент $h$ по $y$, так как не знает секретного ключа
$g^{ab}$.

Теперь опишем собственно криптосистему Эль-Гамаля.
По существу в ней происходит все то же самое, что описано выше,
однако генерация секретного ключа для участников $A$ и $B$ происходит
не заранее, а встроена в сам процесс обмена информацией.

Итак, снова всем участникам переписки известны конечная группа $G$ и
элемент $g \in G$ достаточно большого порядка. Каждый участник 
$A$ загадывает свой натуральное число $a$, которое держит в секрете, и
сообщает всем значение $g^{a}$. Если участник $B$ хочет передать участнику
$A$ элемент $h \in G$, он случайным образом выбирает натуральное $b$ и сообщает
всем пару $(x, y) = (g^b, h(g^{a})^{b})$. По этим данным восстановить элемент
$h$ может только $A$, и делает он это так: $h = yx^{|G| - a}$.

Важная особенность криптосистемы Эль-Гамаля: если участнику $B$ нужно
передать много сообщений для $A$, то для каждого следующего сообщения он может
использовать новое случайное значение параметра $b$, что многократно повышает
надежность всей системы.

%%%%%%%%%%%%%%%%%%%%%%%%%%%%%%%%%%%%%%%%%%%%
%%%%%%%%%%%%%%%%%%%%%%%%%%%%%%%%%%%%%%%%%%%%
%%%%%%%%%%%%%%%%%%%%%%%%%%%%%%%%%%%%%%%%%%%% 12
\newpage
\mysection
\subsection{Кольца.}
\subsection{Коммутативные кольца.}
\subsection{Обратимые элементы, делители нуля и нильпотенты.}
\subsection{Примеры колец.}
\subsection{Поля.}
\subsection{Критерий того, что кольцо вычетов является полем.}



