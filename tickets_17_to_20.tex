%%%%%%%%%%%%%%%%%%%%%%%%%%%%%%%%%%%%%%%%%%%%
%%%%%%%%%%%%%%%%%%%%%%%%%%%%%%%%%%%%%%%%%%%%
%%%%%%%%%%%%%%%%%%%%%%%%%%%%%%%%%%%%%%%%%%%% 17
\mysection
\subsection{Критерий того, что факторкольцо \texorpdfstring{$K[x]/(h)$}{Lg} является полем.}
\subsection{Базис и размерность факторкольца \texorpdfstring{$K[x]/(h)$}{Lg} как векторного
            пространства над полем \texorpdfstring{$K$}{Lg}.}

%%%%%%%%%%%%%%%%%%%%%%%%%%%%%%%%%%%%%%%%%%%%
%%%%%%%%%%%%%%%%%%%%%%%%%%%%%%%%%%%%%%%%%%%%
%%%%%%%%%%%%%%%%%%%%%%%%%%%%%%%%%%%%%%%%%%%% 18
\newpage
\mysection
\subsection{Лексикографический порядок на множестве одночленов
            от нескольких переменных.}
\subsection{Лемма о конечности убывающих цепочек одночленов.}

%%%%%%%%%%%%%%%%%%%%%%%%%%%%%%%%%%%%%%%%%%%%
%%%%%%%%%%%%%%%%%%%%%%%%%%%%%%%%%%%%%%%%%%%%
%%%%%%%%%%%%%%%%%%%%%%%%%%%%%%%%%%%%%%%%%%%% 19
\newpage
\mysection
\subsection{Старший член многочлена от нескольких переменных.}
\subsection{Элементарная редукция многочлена относительно другого многочлена.}
\subsection{Лемма о конечности цепочек элементарных
            редукций относительно системы многочленов.}

%%%%%%%%%%%%%%%%%%%%%%%%%%%%%%%%%%%%%%%%%%%%
%%%%%%%%%%%%%%%%%%%%%%%%%%%%%%%%%%%%%%%%%%%%
%%%%%%%%%%%%%%%%%%%%%%%%%%%%%%%%%%%%%%%%%%%% 20
\newpage
\mysection
\subsection{Остаток многочлена относительно заданной системы многочленов.}
\subsection{Система Грёбнера.}
\subsection{Характеризация систем Грёбнера
            в терминах цепочек элементарных редукций.}


